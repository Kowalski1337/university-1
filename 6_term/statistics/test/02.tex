\documentclass[12pt]{article}

\usepackage{amsmath}
\usepackage{amssymb}
\usepackage{amsthm}
\usepackage{mathtext}
\usepackage[T1,T2A]{fontenc}
\usepackage[utf8]{inputenc}
\usepackage[left=2cm,right=2cm,top=2cm,bottom=2cm,bindingoffset=0cm]{geometry}
\usepackage{geometry}
\usepackage[mathscr]{euscript}
\usepackage{microtype}
\usepackage{enumitem}
\usepackage{bm}
\usepackage{listings}
\usepackage{cancel}
\usepackage{proof}
\usepackage{epigraph}
\usepackage{titlesec}
\usepackage{xcolor}
\usepackage{stmaryrd}
\usepackage[hidelinks]{hyperref}

\usepackage[russian]{babel}
\selectlanguage{russian}

\title{\textsc{Вторая контрольная работа, вариант №2}}
\author{Оганджанян Артем}
\date{}

\begin{document}

\theoremstyle{plain}
\newtheorem*{statement}{Утверждение}

\pagenumbering{gobble}

\maketitle

\section{\texorpdfstring{Задача 2}{Task 2}}

\[
    f_X(x) = \frac{e^{-\left|x\right|}}{2}
\]

\subsection{\texorpdfstring{Характеристическая функция}{Characteristic function}}

\begin{align*}
    \varphi_X(t)
    &= \int_{-\infty}^{+\infty} e^{itx} f_X(x) \mathrm{d}x
    = \int_{-\infty}^{+\infty} e^{itx} \frac{e^{-\left|x\right|}}{2} \mathrm{d}x \\
    &= \frac{1}{2} \left(\int_{-\infty}^0 e^{itx + x} \mathrm{d}x
        + \int_0^{+\infty} e^{itx - x} \mathrm{d}x \right) \\
    &= \frac{1}{2} \left(\left.\frac{1}{it+1} e^{itx+x}\right|_{-\infty}^0
        + \left.\frac{1}{it-1} e^{-x\left(1-it\right)}\right|_0^{+\infty} \right) \\
    &= \frac{1}{2} \left(\frac{1}{it+1} - \frac{1}{it-1} \right)
    = \frac{it - 1 - it - 1}{2(it+1)(it-1)}
    = \frac{- 2}{2(-t^2 - 1)} \\
    &= \frac{1}{t^2 + 1}
\end{align*}

\subsection{\texorpdfstring{Моменты}{Moments}}

\begin{gather*}
    \mathbb E X^n = i^{-n} \left.\frac{\mathrm{d}^n}{\mathrm{d}t^n}\varphi_X(t)\right|_{t=0} \\
    \frac{1}{1 - t} = 1 + t + t^2 + t^3 + \dots \\
    \frac{1}{1 + t^2} = 1 - t^2 + t^4 - t^6 + \dots \\
    \frac{1}{1 + t^2} = \frac{\varphi^{(0)}_X(0)}{0!} t^0 + \frac{\varphi^{(2)}_X(0)}{2!} t^2 + \frac{\varphi^{(4)}_X(0)}{4!} t^4 + \frac{\varphi^{(6)}_X(0)}{6!} t^6 + \dots \\
    \varphi_X^{(n)}(0) = \begin{cases}
        (-1)^k n!, & n = 2k \\
        0,                   & n = 2k + 1
    \end{cases} \\
    \mathbb{E} X^n = \begin{cases}
        n!, & n = 2k \\
        0,  & n = 2k + 1
    \end{cases}
\end{gather*}

\end{document}
