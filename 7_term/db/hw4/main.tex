\documentclass{article}

\usepackage{amsmath}
\usepackage{amssymb}
\usepackage{amsthm}
\usepackage{mathtext}
\usepackage{mathtools}
\usepackage[T1,T2A]{fontenc}
\usepackage[utf8]{inputenc}
\usepackage[left=1cm,right=1cm,top=1cm,bottom=1cm]{geometry}
\usepackage{microtype}
\usepackage{enumitem}
\usepackage{bm}
\usepackage{cancel}
\usepackage{proof}
\usepackage{epigraph}
\usepackage{titlesec}
\usepackage[dvipsnames]{xcolor}
\usepackage{stmaryrd}
\usepackage{cellspace}
\usepackage{cmll}
\usepackage{multirow}
\usepackage{booktabs}
\usepackage{tikz}
\usepackage{caption}
\usepackage{wrapfig}
\usepackage{minted}
\usepackage{svg}

\usepackage[hidelinks]{hyperref}

\usepackage[russian]{babel}
\selectlanguage{russian}

\hypersetup{%
    colorlinks=true,
    linkcolor=blue
}

\DeclareMathOperator{\StudentId}{StudentId}
\DeclareMathOperator{\StudentName}{StudentName}
\DeclareMathOperator{\GroupId}{GroupId}
\DeclareMathOperator{\GroupName}{GroupName}
\DeclareMathOperator{\CourseId}{CourseId}
\DeclareMathOperator{\CourseName}{CourseName}
\DeclareMathOperator{\LecturerId}{LecturerId}
\DeclareMathOperator{\LecturerName}{LecturerName}
\DeclareMathOperator{\Mark}{Mark}

\pagenumbering{gobble}

\title{Домашнее задание №4}
\author{Артем Оганджанян, M3439}
\date{}

\begin{document}

\maketitle

\subsection{\texorpdfstring{Условие}{Statement}}

Дано отношение $R$ со следующими атрибутами:
$\StudentId$,
$\StudentName$,
$\GroupId$,
$\GroupName$,
$\CourseId$,
$\CourseName$,
$\LecturerId$,
$\LecturerName$,
$\Mark$.

Функциональные зависимости:
\begin{align*}
    \StudentId  &\rightarrow \StudentName  & \StudentId            &\rightarrow \GroupId    \\
    \GroupId    &\rightarrow \GroupName    & \CourseId,  \GroupId  &\rightarrow \LecturerId \\
    \CourseId   &\rightarrow \CourseName   & \StudentId, \CourseId &\rightarrow \Mark       \\
    \LecturerId &\rightarrow \LecturerName & &
\end{align*}

Ключ "--- $\{\StudentId, \CourseId\}$.

\subsection{\texorpdfstring{Приведение к пятой нормальной форме}{Task1}}

Для приведения к нормальной форме Бойса-Кодда сначала декомпозируем отношение по
следующим функциональным зависимостям:
$\StudentId  \rightarrow \StudentName$,
$\GroupId    \rightarrow \GroupName$,
$\CourseId   \rightarrow \CourseName$,
$\LecturerId \rightarrow \LecturerName$.
\begin{align*}
    R & = \pi_{\underline \StudentId, \StudentName}(R) \\
      & \bowtie \pi_{\underline \GroupId, \GroupName}(R) \\
      & \bowtie \pi_{\underline \CourseId, \CourseName}(R) \\
      & \bowtie \pi_{\underline \LecturerId, \LecturerName}(R) \\
      & \bowtie \pi_{\underline \StudentId, \GroupId,
                     \underline \CourseId, \LecturerId, \Mark}(R)
\end{align*}
Последнее отношение декомпозируем по функциональным зависимостям 
$\CourseId, \GroupId \rightarrow \LecturerId$ и
$\StudentId \rightarrow \GroupId$.
\begin{align*}
    \pi_{\underline \StudentId, \GroupId,
         \underline \CourseId, \LecturerId, \Mark}(R) &=
    \pi_{\underline \CourseId, \underline{\smash\GroupId}, \LecturerId}(R)\\
    &\bowtie \pi_{\underline \StudentId, \GroupId}(R) \\
    &\bowtie \pi_{\underline \StudentId, \underline \CourseId, \Mark}(R)
\end{align*}

Проверим, что отношения находятся в четвёртой нормальной форме. Все проекции на
два атрибута находятся в 4НФ. В отношении
$\pi_{\underline \CourseId, \underline{\smash\GroupId}, \LecturerId}(R)$
единственная многозначная зависимость "--- это функциональная зависимость
$\CourseId, \GroupId \rightarrow \LecturerId$.
Для неё свойства
\begin{itemize}
    \item $\forall X \twoheadrightarrow Y|Z, \forall A : X \rightarrow A$
    \item $\forall X \twoheadrightarrow Y|Z: X \text{ "--- надключ}$
\end{itemize}
выполняются. Аналогично проверяется
$\pi_{\underline \StudentId, \underline \CourseId, \Mark}(R)$.

Проверим, что отношения находятся в пятой нормальной форме. В проекциях на три
атрибута зависимостей соединений из трёх элементов нет. Значит, могут быть
зависимости соединений только из двух элементов. Мы уже знаем, что в полученных
отношениях все многозначные зависимости "--- это функциональные зависимости.
Поскольку мы уже привели отношения к нормальной форме Бойса-Кодда, все их левые
части "--- это надключи. Значит, по теореме Фейгина, все зависимости соединений
состоят только из надключей.

\begin{figure}[ht]
    \centering
    \includesvg[width=0.55\textwidth]{../ERM}
    \caption{Модель сущность-связь.
    В ассоциации Teaching используются ограничения по Чену.}
\end{figure}

\begin{figure}[ht]
    \centering
    \includesvg[width=0.7\textwidth]{../PDM}
    \caption{Физическая модель.}
\end{figure}

\subsection{\texorpdfstring{Создание базы данных}{Creation script}}
\inputminted{sql}{../scripts/create.sql}

\subsection{\texorpdfstring{Наполнение базы данными}{Inflation script}}
\inputminted{sql}{../scripts/inflate.sql}
 
\end{document}
