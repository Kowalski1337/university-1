\documentclass{article}

\usepackage{amsmath}
\usepackage{amssymb}
\usepackage{amsthm}
\usepackage{mathtext}
\usepackage{mathtools}
\usepackage[T1,T2A]{fontenc}
\usepackage[utf8]{inputenc}
\usepackage[margin=1cm]{geometry}
\usepackage{microtype}
\usepackage{enumitem}
\usepackage{bm}
\usepackage{cancel}
\usepackage{proof}
\usepackage{epigraph}
\usepackage{titlesec}
\usepackage[dvipsnames]{xcolor}
\usepackage{stmaryrd}
\usepackage{cellspace}
\usepackage{cmll}
\usepackage{multirow}
\usepackage{booktabs}
\usepackage{tikz}
\usepackage{caption}
\usepackage{wrapfig}
\usepackage{minted}
\usepackage{svg}

\usepackage[hidelinks]{hyperref}

\usepackage[russian]{babel}
\selectlanguage{russian}

\hypersetup{%
    colorlinks=true,
    linkcolor=blue
}

\DeclareMathOperator{\avg}{avg}

\newcommand{\Students}{\mathrm{Students}}
\newcommand{\StudentId}{\mathrm{StudentId}}
\newcommand{\StudentName}{\mathrm{StudentName}}
\newcommand{\Groups}{\mathrm{Groups}}
\newcommand{\GroupId}{\mathrm{GroupId}}
\newcommand{\GroupName}{\mathrm{GroupName}}
\newcommand{\Courses}{\mathrm{Courses}}
\newcommand{\CourseId}{\mathrm{CourseId}}
\newcommand{\CourseName}{\mathrm{CourseName}}
\newcommand{\Plan}{\mathrm{Plan}}
\newcommand{\Lecturers}{\mathrm{Lecturers}}
\newcommand{\LecturerId}{\mathrm{LecturerId}}
\newcommand{\LecturerName}{\mathrm{LecturerName}}
\newcommand{\Marks}{\mathrm{Marks}}
\newcommand{\Mark}{\mathrm{Mark}}

\newcommand{\join}{\mathbin{\mathrlap{\ltimes}{\rtimes}}}

\pagenumbering{gobble}

\title{Домашнее задание №5}
\author{Артем Оганджанян, M3439}
\date{}

\begin{document}

\renewcommand{\count}{\operatorname{count}}

\maketitle

\subsection{\texorpdfstring{База данных}{DB}}

\begin{itemize}
    \item $\Students: \StudentId, \StudentName, \GroupId$
    \item $\Groups: \GroupId, \GroupName$
    \item $\Courses: \CourseId, \CourseName$
    \item $\Lecturers: \LecturerId, \LecturerName$
    \item $\Plan: \GroupId, \CourseId, \LecturerId$
    \item $\Marks: \StudentId, \CourseId, \Mark$
\end{itemize}

\subsection{\texorpdfstring{Запросы}{Queries}}

\setlist[enumerate]{label*=\arabic*.}
\begin{enumerate}
    \item Информация о студентах с заданной оценкой $m_0$ по предмету «Базы данных».
    \[
        \sigma_{\Mark = m_0}
        (\sigma_{\CourseName = \text{`Базы данных'}}(\Courses)
        \rtimes \Marks)
        \rtimes \Students
    \]
    \item Информация о студентах, не имеющих оценки по предмету «Базы данных».
    \begin{enumerate}
        \item Среди всех студентов.
        \[
            \Students \setminus (
            (\sigma_{\CourseName = \text{`Базы данных'}}(\Courses)
            \rtimes \Marks)
            \rtimes \Students)
        \]
        \item Среди студентов, у которых есть этот предмет.
        \[
            \Students \ltimes (
            \Plan \ltimes
            \sigma_{\CourseName = \text{`Базы данных'}}(\Courses))
            \setminus (
            (\sigma_{\CourseName = \text{`Базы данных'}}(\Courses)
            \rtimes \Marks)
            \rtimes \Students)
        \]
    \end{enumerate}
    \item Информация о студентах, имеющих хотя бы одну оценку у заданного
        лектора с $\LecturerId=l_0$.
    \[
        \Students \ltimes (\sigma_{\LecturerId = l_0}(\Plan) \join \Marks)
    \]
    \item Идентификаторы студентов, не имеющих ни одной оценки у заданного
        лектора с $\LecturerId=l_0$.
    \[
        \pi_{\StudentId}(\Students)\setminus
        \pi_{\StudentId}(\sigma_{\LecturerId = l_0}(\Plan) \join \Marks)
    \]
    \item Студентов, имеющих оценки по всем предметам заданного
        лектора с $\LecturerId=l_0$.
    \[
        \Students \ltimes (
        \pi_{\Students, \CourseId}(\Marks) \div
        \pi_{\CourseId}(\sigma_{\LecturerId=l_0}(\Plan)))
    \]
    \item Для каждого студента имя и предметы, которые он должен посещать.
    \[
        \pi_{\StudentName, \CourseName}(\Students \join \Plan \join \Courses)
    \]
    \item По лектору с $\LecturerId=l_0$ всех студентов,
        у которых он хоть что-нибудь преподавал.
    \[
        \sigma_{\LecturerId=l_0}(\Plan) \rtimes \Students
    \]
    \item Такие пары студентов $\mathrm{Pairs}$,
        что все сданные первым студентом предметы сдал и второй студент.
    \begin{gather*}
        \mathrm{Names} = \pi_{\StudentId, \StudentName}(\Students) \\
        \mathrm{Names1} =
            \rho_{\StudentId=\mathrm{StudentId1}}(
            \rho_{\StudentName=\mathrm{StudentName1}}(
            \mathrm{Names})) \\
        \mathrm{Names2} =
            \rho_{\StudentId=\mathrm{StudentId2}}(
            \rho_{\StudentName=\mathrm{StudentName2}}(
            \mathrm{Names})) \\
        \mathrm{A} =
            \sigma_{\mathrm{StudentId1} \neq \mathrm{StudentId2}}(
            \rho_{\StudentId=\mathrm{StudentId2}}(\pi_{\StudentId, \CourseId}(\Marks))
            \divideontimes
            \rho_{\StudentId=\mathrm{StudentId1}}(\pi_{\CourseId, \StudentId}(\Marks))) \\
        \mathrm{Pairs} = \pi_{\mathrm{StudentId1},\mathrm{StudentName1},
             \mathrm{StudentId2},\mathrm{StudentName2}}(
        \mathrm{Names1} \join \mathrm{A} \join \mathrm{Names2}
        )
    \end{gather*}
    \item Такие группы и предметы, что все студенты группы сдали предмет.
    \[
        (\pi_{\CourseId, \StudentId}(\Marks) \divideontimes
        \pi_{\StudentId, \GroupId}(\Students))
        \join \Groups \join \Courses
    \]
    \item Средний балл студента.
    \begin{enumerate}
        \item По идентификатору $\StudentId=s_0$.
        \[
            \avg_{\Mark, \varnothing}(\sigma_{\StudentId=s_0}(\Marks))
        \]
        \item Для каждого студента.
        \[
            \avg_{\Mark, \{\StudentId\}}(\Marks)
            \join \pi_{\StudentId, \StudentName}(\Students)
        \]
    \end{enumerate}
    \item Средний балл средних баллов студентов каждой группы.
    \[
        \avg_{\Mark, \{\GroupId\}}(
        \avg_{\Mark, \{\StudentId\}}(\Marks)
        \join \pi_{\StudentId, \GroupId}(\Students))
    \]
    \item Для каждого студента число предметов, которые у него были, число сданных предметов и число не сданных предметов.
    \begin{gather*}
        \mathrm{TakenCourses} = 
            \rho_{\CourseId=\mathrm{Taken}}(
            \count_{\CourseId, \{\StudentId\}}(\Students \join \Plan))\\
        \mathrm{PassedCourses} =
            \rho_{\CourseId=\mathrm{Passed}}(
            \count_{\CourseId, \{\StudentId\}}(\Marks))\\
        \mathrm{Summary} = \varepsilon_{
                \mathrm{Failed}=\mathrm{Taken}-\mathrm{Passed}}(
            \mathrm{TakenCourses}
            \join \mathrm{PassedCourses})
    \end{gather*}
\end{enumerate}

\end{document}
