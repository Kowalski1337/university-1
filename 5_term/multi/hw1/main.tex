\documentclass{article}

\usepackage{amsmath}
\usepackage{amssymb}
\usepackage{amsthm}
\usepackage{mathtext}
\usepackage{mathtools}
\usepackage[T1,T2A]{fontenc}
\usepackage[utf8]{inputenc}
%\usepackage{geometry}
\usepackage[left=2cm,right=2cm,top=2cm,bottom=2cm]{geometry}
\usepackage{microtype}
\usepackage{enumitem}
\usepackage{bm}
%\usepackage{listings}
\usepackage{cancel}
\usepackage{proof}
\usepackage{epigraph}
\usepackage{titlesec}
\usepackage[dvipsnames]{xcolor}
\usepackage{stmaryrd}
\usepackage{cellspace}
\usepackage{cmll}
\usepackage{multirow}
\usepackage{booktabs}
\usepackage{tikz}
\usepackage{caption}
\usepackage{wrapfig}
%\usepackage{eucal}

\usepackage[hidelinks]{hyperref}

%\setmainfont[Ligatures=TeX,SmallCapsFont={Times New Roman}]{Palatino Linotype}

\usepackage[russian]{babel}
\selectlanguage{russian}

\hypersetup{%
    colorlinks=true,
    linkcolor=blue
}

\pagenumbering{gobble}

\title{}
\author{Артем Оганджанян}
\date{}

\begin{document}

\usetikzlibrary{arrows.meta,positioning,matrix}

\maketitle

\begin{figure}[h]
    \centering
    \begin{tikzpicture}[->,>=Stealth,every path/.style={draw=black,thick}]
    \begin{scope}[every node/.style={circle,thick,draw},align=center]
        \node (A1100) at (0,0) {P1, Q1 \\ $a = 0$ \\ $b = 0$};
        \node (A2110) at (0,-3) {P2, Q1 \\ $a = 1$ \\ $b = 0$};
        \node (A3110) at (0,-6) {P3, Q1 \\ $a = 1$ \\ $b = 0$};
        \node (A4110) at (0,-9) {P4, Q1 \\ $a = 1$ \\ $b = 0$};

        \node (A1201) at (3,0) {P1, Q2 \\ $a = 0$ \\ $b = 1$};
        \node (A2211) at (3,-3) {P2, Q2 \\ $a = 1$ \\ $b = 1$};
        \node (A3211) at (3,-6) {P3, Q2 \\ $a = 1$ \\ $b = 1$};
        \node (A4211) at (3,-9) {P4, Q2 \\ $a = 1$ \\ $b = 1$};

        \node (A1301) at (6,0) {P1, Q3 \\ $a = 0$ \\ $b = 1$};
        \node (A2311) at (6,-3) {P2, Q3 \\ $a = 1$ \\ $b = 1$};
        \node (A3311) at (6,-6) {P3, Q3 \\ $a = 1$ \\ $b = 1$};
        \node (A4311) at (6,-9) {P4, Q3 \\ $a = 1$ \\ $b = 1$};

        \node (A1s01) at (9,0) {P1, Qs \\ $a = 0$ \\ $b = 1$};
        \node (A2s11) at (9,-3) {P2, Qs \\ $a = 1$ \\ $b = 1$};
        \node (A3s11) at (9,-6) {P3, Qs \\ $a = 1$ \\ $b = 1$};
        \node[draw=none] (A4s11) at (9,-9) {\phantom{P4, Qs} \\ \phantom{$a = 1$} \\ \phantom{$b = 1$}};
    \end{scope}

    \path [->] (A1100) edge (A2110)
                       edge (A1201)
               (A2110) edge (A2211)
                       edge (A3110)
               (A3110) edge (A4110)
                       edge (A3211)
               (A4110) edge[bend left] (A1100)
                       edge (A4211)

               (A1201) edge (A2211)
                       edge[bend left=40] (A1s01)
               (A2211) edge (A2311)
               (A3211) edge (A4211)
                       edge (A3311)
                       edge[bend left=40, dashed, color=red] (A3s11)
               (A4211) edge (A4311)
                       edge[bend left] (A1201)

               (A1301) edge (A2311)
                       edge[bend left=40] (A1100)
               (A2311) edge[bend left=40] (A2110)
               (A3311) edge (A4311)
                       edge[bend left=40] (A3110)
               (A4311) edge[bend left] (A1301)
                       edge[bend left=40] (A4110)
                       
               (A1s01) edge (A2s11)
               (A2s11) edge[dashed, color=red] (A3s11)
                       edge[draw=none, bend left] (A4s11);

    \draw (A2211) to [out=60, in=30, looseness=5] (A2211);
    \draw (A2311) to [out=60, in=30, looseness=5] (A2311);
    \draw (A2s11) to [out=60, in=30, looseness=5] (A2s11);

    \end{tikzpicture}
    \caption{Диаграмма.}
    \label{diagram}
\end{figure}

\begin{enumerate}
    \item Рисунок \ref{diagram}.
    \item Всего 14 достижимых состояний.
          Это гораздо меньше чем 64, потому что в каждом из состояний потоков возможно только одно значение переменных.
    \item В состояние Qs поток Q мог перейти только из состояния Q2.
          Переменная $b$ меняется только в потоке Q, и при переходе из состояния Q1 в состояние Q2 её значение меняется на 1.
          Значит, в состоянии Qs значение $b$ равно $1$. Аналогично, в состоянии P3 значение $a$ равно $1$.
          В рассматриваемое состояние системы можно непосредственно перейти либо из пары состояний (P2, Qs),
          либо из пары состояний (P3, Q2).
          Но из состояния P2 при $b=1$ нельзя перейти в состояние P3, а из состояния Q2 при $a=1$ нельзя перейти в состояние Qs.
          Значит, такая ситуация невозможна (рисунок \ref{diagram}).
    \item Допустим обратное, то есть существует такое состояние системы, из которого состояние Qs не достижимо.
          Тогда из состояния Q2 можно будет перейти только в состояние Q3.
          Значит, из любого состояния потока Q можно перейти в состояние Q1, в котором обязательно $b=0$.
          При $b=0$ можно перейти в любое состояние потока P, в частности, можно перейти в состояние P1, в котором обязательно $a=0$.
          Но при $a=0$ из состояния Q2 можно перейти в состояние Qs, то есть Qs достижимо. Противоречие.
\end{enumerate}

\end{document}
