\documentclass{article}

\usepackage{amsmath}
\usepackage{amssymb}
\usepackage{amsthm}
\usepackage{mathtext}
\usepackage[T1,T2A]{fontenc}
\usepackage[utf8]{inputenc}
\usepackage[mathscr]{euscript}
\usepackage{microtype}
\usepackage{enumitem}
\usepackage{bm}
\usepackage{listings}
\usepackage{cancel}
\usepackage{proof}
\usepackage{epigraph}
\usepackage{titlesec}
\usepackage{xcolor}
\usepackage{stmaryrd}
\usepackage[hidelinks]{hyperref}
\usepackage{mathtools}
\usepackage{pgfplots}
\usepackage[left=3cm,right=3cm,top=2cm,bottom=2cm]{geometry}
\usepackage{multicol}
\usepackage{subfig}
\pgfplotsset{compat=1.13}
\DeclarePairedDelimiter{\floor}{\lfloor}{\rfloor}

\usepackage[russian]{babel}
\selectlanguage{russian}

\title{Домашнее задание \textnumero 3}
\author{Артем Оганджанян}
\date{}

\begin{document}

\maketitle

\begin{multicols}{2}
\begin{verbatim}
$ lscpu                                                                        
Architecture:          x86_64
CPU op-mode(s):        32-bit, 64-bit
Byte Order:            Little Endian
CPU(s):                4
On-line CPU(s) list:   0-3
Thread(s) per core:    2
Core(s) per socket:    2
Socket(s):             1
NUMA node(s):          1
Vendor ID:             GenuineIntel
CPU family:            6
Model:                 42
Model name:            Intel(R) Core(TM)
                       i5-2450M CPU
                       @ 2.50GHz
Stepping:              7
CPU MHz:               957.714
CPU max MHz:           3100,0000
CPU min MHz:           800,0000
BogoMIPS:              4989.14
Virtualization:        VT-x
L1d cache:             32K
L1i cache:             32K
L2 cache:              256K
L3 cache:              3072K
NUMA node0 CPU(s):     0-3
\end{verbatim}
\vfill
\columnbreak
\begin{verbatim}
$ sudo dmidecode --type 17                                                                                                                    
# dmidecode 3.0
Getting SMBIOS data from sysfs.
SMBIOS 2.7 present.

Handle 0x0006, DMI type 17, 34 bytes
Memory Device
	Array Handle: 0x0005
	Error Information Handle: Not Provided
	Total Width: 64 bits
	Data Width: 64 bits
	Size: 8192 MB
	Form Factor: SODIMM
	Set: None
	Locator: Bottom-Slot 1(top)
	Bank Locator: BANK 0
	Type: DDR3
	Type Detail: Synchronous
	Speed: 1600 MHz
	Manufacturer: Kingston
	Serial Number: 61245868
	Asset Tag: 9876543210
	Part Number: 99U5428-063.A00LF 
	Rank: Unknown
	Configured Clock Speed: Unknown

Handle 0x0008, DMI type 17, 34 bytes
Memory Device
	Array Handle: 0x0005
	Error Information Handle: Not Provided
	Total Width: 64 bits
	Data Width: 64 bits
	Size: 4096 MB
	Form Factor: SODIMM
	Set: None
	Locator: Bottom-Slot 2(under)
	Bank Locator: BANK 2
	Type: DDR3
	Type Detail: Synchronous
	Speed: 1600 MHz
	Manufacturer: Samsung
	Serial Number: 00C23072
	Asset Tag: 9876543210
	Part Number: M471B5273DH0-CK0  
	Rank: Unknown
	Configured Clock Speed: Unknown
\end{verbatim}
\end{multicols}

\clearpage

\begin{figure}[ht]
    \hfill
    \subfloat{
        \begin{tikzpicture}
            \begin{axis}[
                    width=7cm,
                    xlabel={Количество потоков},
                    ylabel={Время выполнения операции, нс},
                    ymin=0,
                    xtick={2,4,6},
                    legend style={at={(0,1)},anchor=south west}
                ]
                \addplot+[color=red,error bars/.cd,y dir=both,y explicit] coordinates {
                    (2, 19.452 ) +- (0,   3.497  )
                    (4, 12.836 ) +- (0,  8.089  )
                    (6, 19.354 ) +- (0,  4.785  )
                };
                \addlegendentry{разделяемая volatile}
                \addplot+[color=blue,error bars/.cd,y dir=both,y explicit] coordinates {
                    (2, 4.910 ) +- (0,   2.341  )
                    (4, 8.444 ) +- (0,  0.544  ) 
                    (6, 9.461 ) +- (0,  1.042  )
                };
                \addlegendentry{разделяемая не volatile}
            \end{axis}
        \end{tikzpicture}
    }
    \hfill
    \subfloat{
        \begin{tikzpicture}
            \begin{axis}[
                    width=7cm,
                    xlabel={Количество потоков},
                    ylabel={Время выполнения операции, нс},
                    ymin=0,
                    xtick={2,4,6},
                    legend style={at={(0,1)},anchor=south west}
                ]
                \addplot+[color=red,error bars/.cd,y dir=both,y explicit] coordinates {
                    (2, 5.955 ) +- (0,   0.675  )
                    (4, 9.841 ) +- (0,  9.380  )
                    (6, 15.277 ) +- (0,  5.852  )
                };
                \addlegendentry{неразделяемая volatile}
                \addplot+[color=blue,error bars/.cd,y dir=both,y explicit] coordinates {
                    (2, 2.896 ) +- (0,   0.612  )
                    (4, 4.447 ) +- (0,  0.284  ) 
                    (6, 6.924 ) +- (0,  0.461  )
                };
                \addlegendentry{неразделяемая не volatile}
            \end{axis}
        \end{tikzpicture}
    }
    \hfill
    \caption{Время доступа для разделяемой и неразделяемой переменной.}
\end{figure}

\begin{figure}[ht]
    \hfill
    \subfloat{
        \begin{tikzpicture}
            \begin{axis}[
                    width=7cm,
                    xlabel={Количество потоков},
                    ylabel={Время выполнения операции, нс},
                    ymin=0,
                    xtick={2,4,6},
                    legend style={at={(0,1)},anchor=south west}
                ]
                \addplot+[color=red,error bars/.cd,y dir=both,y explicit] coordinates {
                    (2, 14.124 ) +- (0,   2.853  )
                    (4, 11.731 ) +- (0,  3.808  )
                    (6, 14.630 ) +- (0,  3.834  )
                };
                \addlegendentry{чтение volatile}
                \addplot+[color=blue,error bars/.cd,y dir=both,y explicit] coordinates {
                    (2, 7.490 ) +- (0,   4.726  )
                    (4, 14.454 ) +- (0,  1.160  )
                    (6, 15.007 ) +- (0,  2.388  )
                };
                \addlegendentry{чтение не volatile}
            \end{axis}
        \end{tikzpicture}
    }
    \hfill
    \subfloat{
        \begin{tikzpicture}
            \begin{axis}[
                    width=7cm,
                    xlabel={Количество потоков},
                    ylabel={Время выполнения операции, нс},
                    ymin=0,
                    xtick={2,4,6},
                    legend style={at={(0,1)},anchor=south west}
                ]
                \addplot+[color=red,error bars/.cd,y dir=both,y explicit] coordinates {
                    (2, 24.780 ) +- (0,   9.022  )
                    (4, 13.942 ) +- (0, 12.546  )
                    (6, 24.079 ) +- (0,  6.357  )
                };
                \addlegendentry{запись volatile}
                \addplot+[color=blue,error bars/.cd,y dir=both,y explicit] coordinates {
                    (2, 2.330 ) +- (0,   0.064  )
                    (4, 2.434 ) +- (0,  0.090  )
                    (6, 3.916 ) +- (0,  0.455  )
                };
                \addlegendentry{запись не volatile}
            \end{axis}
        \end{tikzpicture}
    }
    \hfill
    \caption{Время доступа для чтения и записи.}
\end{figure}

По понятным причинам работа с не-volatile переменной быстрее.
При возрастании количества потоков логично было бы ожидать увеличение времени доступа, однако на проведённых измерениях это не наблюдается.
Работа с неразделяемыми переменными происходит несколько быстрее.

\clearpage

\begin{figure}[ht]
    \hfill
    \subfloat{
        \begin{tikzpicture}
            \begin{axis}[
                    width=7cm,
                    xlabel={Количество потоков},
                    ylabel={Время выполнения операции, нс},
                    ymin=0,
                    xtick={2,4,6},
                    legend style={at={(0,1)},anchor=south west}
                ]
                \addplot+[color=red,error bars/.cd,y dir=both,y explicit] coordinates {
                    (2, 20.356 ) +- (0,   2.505  )
                    (4, 13.791 ) +- (0,  8.441  )
                    (6, 17.716 ) +- (0,  3.495  )
                };
                \addlegendentry{разделяемая volatile}
                \addplot+[color=blue,error bars/.cd,y dir=both,y explicit] coordinates {
                    (2, 141.801 ) +- (0, 124.256  )
                    (4, 89.089 ) +- (0, 48.309  )
                    (6, 59.327 ) +- (0,  9.314  )
                };
                \addlegendentry{разделяемая synchronized}
                \addplot+[color=green,error bars/.cd,y dir=both,y explicit] coordinates {
                    (2, 232.104 ) +- (0,  78.828  )
                    (4, 170.478 ) +- (0, 46.268  )
                    (6, 92.207 ) +- (0, 10.133  )
                };
                \addlegendentry{разделяемая ReentrantLock}
            \end{axis}
        \end{tikzpicture}
    }
    \hfill
    \subfloat{
        \begin{tikzpicture}
            \begin{axis}[
                    width=7cm,
                    xlabel={Количество потоков},
                    ylabel={Время выполнения операции, нс},
                    ymin=0,
                    xtick={2,4,6},
                    legend style={at={(0,1)},anchor=south west}
                ]
                \addplot+[color=red,error bars/.cd,y dir=both,y explicit] coordinates {
                    (2, 5.595 ) +- (0,   0.108  )
                    (4, 8.640 ) +- (0,  6.966  )
                    (6, 13.069 ) +- (0,  2.439  )
                };
                \addlegendentry{неразделяемая volatile}
                \addplot+[color=blue,error bars/.cd,y dir=both,y explicit] coordinates {
                    (2, 30.799 ) +- (0,   3.174  )
                    (4, 40.055 ) +- (0, 12.537  )
                    (6, 58.341 ) +- (0,  8.762  )
                };
                \addlegendentry{неразделяемая synchronized}
                \addplot+[color=green,error bars/.cd,y dir=both,y explicit] coordinates {
                    (2, 26.549 ) +- (0,   1.273  )
                    (4, 37.779 ) +- (0,  2.774  )
                    (6, 62.117 ) +- (0,  3.348  )
                };
                \addlegendentry{неразделяемая ReentrantLock}
            \end{axis}
        \end{tikzpicture}
    }
    \hfill
    \caption{Время доступа при различных методах синхронизации.}
\end{figure}

На графиках видно, что блокировки сильно дороже работы с volatile переменной.
Это объясняется тем, что они предоставляют гораздо больше гарантий, для соблюдения которых нужны дополнительные операции.
Работа с неразделяемыми переменными проиходит быстрее: у каждого потока отдельное состояние,
следовательно при доступе к состоянию потоки не тратят время на ожидание отпускания блокировок другими потоками.

Здесь при возрастании количества потоков наблюдается уменьшение времени доступа, что я объяснить не могу.
\end{document}
